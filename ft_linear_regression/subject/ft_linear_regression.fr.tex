%******************************************************************************%
%                                                                              %
%                  sample.tex for LaTeX                                        %
%                  Created on : Tue Mar 10 13:27:28 2015                       %
%                  Made by : David "Thor" GIRON <thor@42.fr>                   %
%                                                                              %
%******************************************************************************%

\documentclass{42}
\graphicspath{{images/}}

%******************************************************************************%
%                                                                              %
%                                  Prologue                                    %
%                                                                              %
%******************************************************************************%
\begin{document}



                           \title{ft\_linear\_regression}
                          \subtitle{An introduction to machine learning}
                        \member{Baptiste Jacob}{bjacob@student.42.fr}

\summary {
  In this project you are going to implement your first machine learning algorithm.
}

\maketitle

\tableofcontents


%******************************************************************************%
%                                                                              %
%                                  Préambule                                   %
%                                                                              %
%******************************************************************************%
\chapter{Préambule}

	What i think is the best definition for machine learning : \\\\
	``A computer program is said to learn from experience E with respect to some class of tasks T and performance measure P, if its performance at tasks in T, as measured by P, improves with experience E'' \\
	\begin{flushright}Tom M. Mitchell\end{flushright}

%******************************************************************************%
%                                                                              %
%                                 Introduction                                 %
%                                                                              %
%******************************************************************************%
\chapter{Introduction}

	Machine learning is a growing field of computer science that may seem a bit complicated and only reserved to mathematicians. You may have heard of neural network or k-means clustering and don't undersdand how they work or how to code this kind of algorithm. Don't worry on this subject we will start with a basic and very simple machine learning algorithm.

%******************************************************************************%
%                                                                              %
%                                  Objective                                   %
%                                                                              %
%******************************************************************************%
\chapter{Objective}

	The objective of this project is to enable you to understand basic concept of machine learning. In this subject you will have to create a program that predicts the price of a car by using a \href{https://en.wikipedia.org/wiki/Linear\_function}{linear function} train with a \href{https://en.wikipedia.org/wiki/Gradient\_descent}{gradient descent algorithm}.\\ 
	We will be using a precise example in this subject, but at the end you will be able to use this algorithm with any other dataset.

%******************************************************************************%
%                                                                              %
%                            General instructions                              %
%                                                                              %
%******************************************************************************%
\chapter{General instructions}

	In this project you are free to use whatever language you want. If you use C, you have to respect the Norm, as usual.\\\\

	You are also free to use any libraries you want as long as it does not do all the work for you. For example, if you use python numpy.polyfit is considered as a cheat case.\\\\

	\hint {You should use a language that allows you to easily visualize your data it will be very helpful when you will have to debug}

%******************************************************************************%
%                                                                              %
%                            What you have to do                               %
%                                                                              %
%******************************************************************************%
\chapter{What you have to do}
	
	In this project you will realize a simple linear regression with a single feature which is the mileage of the car.\\\\

	To do so you need to create two programs :\\
	\vspace{5mm}
	\begin{itemize}\itemsep6pt
		\item The first program will be use to predict the price of a car for a given mileage. If you launch this program it will prompt you for a mileage and print you the estimated price for this mileage. This program will use the following hypotheses to predict the price :
		\[
		estimatePrice (mileage) = \theta_{0} + ( \theta_{1} * mileage )
		\]
		Before we run our training program theta0 and theta1 will be set to 0.
		\item The second program will be use to train our model. This program will read our dataset file and perform a linear regression in this data.\\
		After performing the linear regression you will save your variable theta0 and theta1 to use it in your first program.\\
		You will be using the following formula :
		\[
		tmp\theta_{0} = learningRate * \frac{1}{m} \sum_{i=0}^{m-1} (estimatePrice(mileage[i]) - price[i])
		\]
		\[
		tmp\theta_{1} = learningRate * \frac{1}{m} \sum_{i=0}^{m-1} (estimatePrice(mileage[i]) - price[i]) * milleage[i]
		\]
		I let you guess what m is :)\\
		note that the estimatePrice is the same as in our first program but it use our temporary computed theta0 and theta1.
		Also don't forget to simultaneously update theta0 and theta1.
	\end{itemize}

%******************************************************************************%
%                                                                              %
%                                    Bonuses                                   %
%                                                                              %
%******************************************************************************%
\chapter{Bonuses}

	\begin{itemize}\itemsep1pt
		\item ploting the data into a graph to see there repartition.
		\item ploting the line that fit the data to see the result of our work.
		\item create a program to calculate the precision of our algorithm.
	\end{itemize}
	\vspace{10mm}
	Any other bonuses that can make your program awesome.


%******************************************************************************%
%                                                                              %
%                                Peer-evaluation                               %
%                                                                              %
%******************************************************************************%
\chapter{Peer-evaluation}

	The peer who will grade you will have to verify :\\
	\begin{itemize}\itemsep1pt
		\item for cheat function
		\item if hypotheses is the same as specified before
		\item if taining function is the same as specified before
	\end{itemize}

\end{document}
%******************************************************************************%
