%%%%%%%%%%%%%%%%%%%%%%%%%%%%%%%%%%%%%%%%%%%%%%%%%%%%%%%%%%%%%%%%%%%%%%%%%%%%%%%%
% test.tex
%%%%%%%%%%%%%%%%%%%%%%%%%%%%%%%%%%%%%%%%%%%%%%%%%%%%%%%%%%%%%%%%%%%%%%%%%%%%%%%%
%
% Authors:
% - pvincent
% - rdavid
%
% Contributors:
% - Unknown for now
%
%%%%%%%%%%%%%%%%%%%%%%%%%%%%%%%%%%%%%%%%%%%%%%%%%%%%%%%%%%%%%%%%%%%%%%%%%%%%%%%%



\documentclass{42}



%%%%%%%%%%%%%%%%%%%%%%%%%%%%%%%%%%%%%%%%%%%%%%%%%%%%%%%%%%%%%%%%%%%%%%%%%%%%%%%%
% Prologue
%%%%%%%%%%%%%%%%%%%%%%%%%%%%%%%%%%%%%%%%%%%%%%%%%%%%%%%%%%%%%%%%%%%%%%%%%%%%%%%%

\begin{document}

%Table des matieres
\title{OpenGL Project}
\subtitle{HumanGL}

\member {42 staff}{staff@42.fr}

\summary
{
	This project is an introduction to OpenGL.
}

\maketitle

\tableofcontents

% Valeurs utilisees pour la generation de headers d'exercices
\turnindir{svn+ssh://rendus@rendus.42.fr/sujetdetest-2142-login\_x}

\newpage
%%%%%%%%%%%%%%%%%%%%%%%%%%%%%%%%%%%%%%%%%%%%%%%%%%%%%%%%%%%%%%%%%%%%%%%%%%%%%%%%
% Start document
%%%%%%%%%%%%%%%%%%%%%%%%%%%%%%%%%%%%%%%%%%%%%%%%%%%%%%%%%%%%%%%%%%%%%%%%%%%%%%%%
\chapter{Foreword}
{
\small
\noindent
Do you remember the 21st night of September?\\
Love was changing the mind of pretenders\\
While chasing the clouds away\\
\\
Our hearts were ringing\\
In the key that our souls were singing.\\
As we danced in the night,\\
Remember - how the stars stole the night away, yeah yeah yeah.\\
\\
Hey hey hey,\\
Ba de ya - say do you remember\\
Ba de ya - dancing in September\\
Ba de ya - never was a cloudy day\\
\\
Ba duda, ba duda, ba duda, badu\\
Ba duda, badu, ba duda, badu\\
Ba duda, badu, ba duda\\
\\
My thoughts are with you\\
Holding hands with your heart to see you\\
Only blue talk and love,\\
Remember - how we knew love was here to stay\\
\\
Now December found the love that we shared in September.\\
Only blue talk and love,\\
Remember - the true love we share today\\
Hey hey hey\\
Ba de ya - say do you remember\\
Ba de ya - dancing in September\\
Ba de ya - never was a cloudy day....there was a\\
Ba de ya - say do you remember\\
Ba de ya - dancing in September\\
Ba de ya - golden dreams were shiny days\\
\\
Now our bell was ringing, aha\\
Our souls were singing.\\
Do you remember every cloudy day - yau !\\
\\
There was a\\
Ba de ya - say do you remember\\
Ba de ya - dancing in September\\
Ba de ya - never was a cloudy day....there was a\\
Ba de ya - say do you remember\\
Ba de ya - dancing in September\\
Ba de ya - golden dreams were shiny days\\
\\
Ba de ya de ya de ya\\
Ba de ya de ya de ya\\
Ba de ya de ya de ya - De ya.... {x2}\\

This subject won't be easier if you are listening Disco, but that's freakin' cool.\\
And if you're feeling bad about some difficulties, just think about Travolta.
}

\chapter{The project}
\section{What you're gonna do}

In this project, you must implement a skeletal animation with a hierarchical model.\\
Body parts are correctly articulated using the OpenGL matrix stack. If the torso rotate, all the members must follow logically, therefore if the upper arm move only the forearm have to follow. When you modify the size of a member, related parts automatically reposition themself.

Your model will have the following parts :\\
\begin{itemize}
	\item a head
	\item a torso
	\item two arms with
	\begin{itemize}
		\item upper arm
		\item forearm
	\end{itemize}
	\item two legs with
	\begin{itemize}
		\item thigh
		\item lower part
	\end{itemize}
\end{itemize}
And it should be able to walk, jump and stay put.

\section{Constraints}

\subsection{Realisation}
You should call the same drawing function for each body part and only one time per part.\\
This function will draw a 1x1x1 geometric shape in the current matrix.

\info
{
	Upper and lower part of the same member are indeed two different parts.
}

\subsection{Language}

A makefile or something similar is required.

You can use the graphic library of your choice (SDL2, Glut, SFML..) but you need to use OpenGL matrices and associated functions.

You are free to use whatever language you want. If you use C, you have to respect the Norm, as usual.

\section{Bonuses}

When your hierarchical model is completely working, it would be easy to add :
\begin{itemize}
	\item More body parts or a completely different model.
	\item Other move patterns (Disco dance, Kung-fu fighting, etc etc..)
	\item A kick-ass interface where you can for example modify body part size.
	\item Camera gestion.
\end{itemize}

\section{Defense sessions}

Be prepared to :
\begin{itemize}
	\item Obviously run the program and show the different move patterns.
	\item Change member sizes.
	\item Show your drawing function, his calls and explain how it works.
	\item Explain your hierarchical model and the resulting matrix stack.
\end{itemize}

\chapter{Illustrations}

% \section{Section premiere}

% \subsection{Sous-section the first}

% \begin{itemize}
% 	\item Liste
% 	\item de
% 	\item trucs
% 	\begin{itemize}
% 		\item nested
% 		\item trop bien
% 	\end{itemize}
% 	\item mais j'aime les poneys
% \end{itemize}


% \begin{figure}[ht!]
%   \centering
%   \includegraphics[width=90mm]{42.jpg}
% \end{figure}


% \newpage

% \begin{42ccode}
% /* Awesome code is awesome */
% int		main(int ac, char** av)
% {
% 	my_putstr("J'aime le 101010");
%         return 0;
% }
% \end{42ccode}

% \begin{42console}
% $>ls
% Makefile  test.aux  test.log  test.outtest.pdf  test.tex  test.toc
% $>
% \end{42console}


% \hint
% {
% 	Hint box
% }

% \warn
% {
% 	Warn box
% }

% \info
% {
% 	Info box
% }

% \chapter{Ceci est un exercice}

% \extitle{Ceci est le titre de l'exercice}
% \exnumber{00}
% \exscore{42}
% \exfiles{tamere.h,tamere.c}

% \makeheaderbasic

% Ceci est le sujet de l'exercice.

%%%%%%%%%%%%%%%%%%%%%%%%%%%%%%%%%%%%%%%%%%%%%%%%%%%%%%%%%%%%%%%%%%%%%%%%%%%%%%%%
% End document
%%%%%%%%%%%%%%%%%%%%%%%%%%%%%%%%%%%%%%%%%%%%%%%%%%%%%%%%%%%%%%%%%%%%%%%%%%%%%%%%
\end{document}
