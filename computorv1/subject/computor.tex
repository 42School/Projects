%******************************************************************************%
%                                                                              %
%              computor.tex for peer-pedagogy                                  %
%              Created on : Wed Mar 18 12:09:18 2015                           %
%              Made by : Sebastien Julliot <sjulliot@student.42.fr>            %
%                                                                              %
%******************************************************************************%
\documentclass{42}
\usepackage{amsmath, amsfonts, amssymb}
\graphicspath{{images/}}



%******************************************************************************%
%                                                                              %
%                                   Prologue                                   %
%                                                                              %
%******************************************************************************%
\begin{document}

                              \title{Computor v1}
                     \subtitle{"Moi non plus j'ai pas le bac"}
                \member{Sebastien Julliot}{sjulliot@student.42.fr}
                     \member{42 staff}{pedago@staff.42.fr}

\summary {
  Ce projet est le premier d'une série ayant pour but de vous
  faire renouer avec les maths, qui vous seront très utiles -voire
  nécessaires- pour de nombreux autres projets.
}

\maketitle

\tableofcontents


%******************************************************************************%
%                                                                              %
%                                  Preambule                                   %
%                                                                              %
%******************************************************************************%
\chapter{Préambule}

    Un polynôme est une expression formelle de la forme:\\

	\begin{equation}
        P(X)=\sum_{k=0}^{n} a_k X^k
    \end{equation}\\

	Où X est appelé indéterminée du polynôme.\\

	Le produit de deux polynômes est ainsi défini par\\

	\begin{equation}
        \left(\sum_{i=0}^n a_iX^i\right)\left(\sum_{j=0}^m b_jX^j\right) = \sum_{k=0}^{n+m} \left(\sum_{i+j = k}a_ib_j\right)X^k.
	\end{equation}

    \begin{center}
        \includegraphics[scale=0.35]{what1}
	    \includegraphics[scale=0.35]{what2}
	    \includegraphics[scale=0.35]{what3}
	    \includegraphics[scale=0.35]{what4}
	    \includegraphics[scale=0.35]{what5}
	    \includegraphics[scale=0.35]{what6}
	    \includegraphics[scale=0.35]{what7}
	    \includegraphics[scale=0.35]{what8}
    \end{center}

    \hint {
      La video offre une explication plus... comprehensible.
    }



%******************************************************************************%
%                                                                              %
%                                 Introduction                                 %
%                                                                              %
%******************************************************************************%
\chapter{Introduction}

    Le but de ce sujet est de vous faire coder un programme qui résout
    des équations simples. Le programme prendra en paramètre une
    équation polynomiale. C'est-à-dire ne faisant intervenir que des
    puissances, aucune fonction compliquée. Le programme devra
    afficher sa (ses) solution(s).\\

    Pourquoi des polynômes? Parce que c'est l'un des outils
    mathématiques les plus simples et les plus puissants. On s'en sert
    dans tous les domaines et à tous les niveaux pour simplifier et
    exprimer beaucoup de choses. Par exemple, les fonctions
    \texttt{sin}, \texttt{cos}, et \texttt{tan} sont calculées à
    l'aide de polynômes.

    \info {
      En fait, il existe même un résultat : le théorème de
      \texttt{Stone-Weierstrass}, qui dit que toutes les fonctions
      "courantes", (celles qui sont bien lisses et jolies), peuvent
      être exprimées à l'aide de polynômes.
    }



%******************************************************************************%
%                                                                              %
%                                  Objectifs                                   %
%                                                                              %
%******************************************************************************%
\chapter{Objectifs}

    L’idée est de vous faire (re)prendre contact avec la manipulation
    d'outils mathématiques élémentaires, qui pourront vous être utiles
    dans de nombreux autres sujets de 42. Il ne s'agit donc pas de
    "faire des maths pour faire des maths", mais bien de vous
    permettre d'aborder plus progressivement et sereinement les sujets
    faisant intervenir des maths.\\

    Voici une liste non exhaustive des sujets où savoir ce que sont et
    comment manipuler les polynômes pourrait bien vous être utile:\\

	\begin{itemize}\itemsep1pt
        \item Fractol
	    \item RT
	    \item mod1
	    \item Expert System
	    \item Infin Mult\\
    \end{itemize}

    Par ailleurs, ce petit sujet sera complété par d'autres sur des
    sujets variés, pour comprendre ce que vous faites plutôt que
    simplement copier une formule sur Internet.


%******************************************************************************%
%                                                                              %
%                              Partie obligatoire                              %
%                                                                              %
%******************************************************************************%
\chapter{Partie obligatoire}

    Ecrivez un programme qui résout une équation polynomiale de
    degr\'e inf\'erieur ou e\'gal \`a 2. Vous devrez afficher au
    moins:\\

    \begin{itemize}\itemsep1pt
        \item La forme r\'eduite de l'\'equation.
        \item Le degr\'e de l'\'equation.
        \item Sa ou ses solution(s), ainsi que le signe du discrimiant
          quand cela a du sens.\\
    \end{itemize}

      Exemples:\\

    \begin{42console}
$>./computor "5 * X^0 + 4 * X^1 - 9.3 * X^2 = 1 * X^0"
Reduced form: 4 * X^0 + 4 * X^1 - 9.3 * X^2 = 0
Polynomial degree: 2
Discriminant is strictly positive, the two solutions are:
0.905239
0.475131
$>./computor "5 * X^0 + 4 * X^1 = 4 * X^0"
Reduced form: 1 * X^0 + 4 * X^1 = 0
Polynomial degree: 1
The solution is:
-0.25
./computor "8 * X^0 - 6 * X^1 + 0 * X^2 - 5.6 * X^3 = 3 * X^0"
Reduced form: 5 * X^0 - 6 * X^1 + 0 * X^2 - 5.6 * X^3 = 0
Polynomial degree: 3
The polynomial degree is stricly greater than 2, I can't solve.\end{42console}

    On considèrera toujours que l'entrée est bien formatée (ie. tous
    les termes sont de la forme $a * x^p$. Les puissances sont bien ordonnées et toutes
    présentes.

    \info {
	  La résolution des équations de degré trois ou plus n'est pas
      demandée. Ca ferait un excellent nouveau sujet non ? :)
    }



%******************************************************************************%
%                                                                              %
%                                 Partie Bonus                                 %
%                                                                              %
%******************************************************************************%
\chapter{Partie bonus}

    Voici une liste de bonus qu'il pourrait être utile d'implémenter:\\

    \begin{itemize}

        \item Gérer les erreurs sur l'entrée (lexique et syntaxe).

	    \item Gérer les entrées sorties sous forme naturelle.
        \begin{42console}
./computor "5 + 4 * X + X^2= X^2"
Reduced form: 5 + 4 * X = 0
Polynomial degree: 1
The solution is:
-1.25\end{42console}

        \item Afficher la (les) solution sous forme de fraction
          irréductible quand c'est intéressant.

	    \item Afficher des étapes intermédiaires

	    \item ...

    \end{itemize}

%******************************************************************************%
%                                                                              %
%                                  Consignes                                   %
%                                                                              %
%******************************************************************************%

\chapter{Consignes}

	\begin{itemize}

		\item Pensez aux solutions complexes quand le degré vaut 2 ;)

		\item Le choix du langage est à votre discrétion.

		\item Cela dit, vous n'avez évidemment droit à aucune
          fonction/biblioth\`eque mathématique (hors addition et
          multiplication de réels) que vous n'ayez pas implémentée
          vous-mêmes.

		\item Si vous travaillez dans un langage compilable (C/C++
          notamment), vous rendrez un Makefile contenant les règles
          habituelles.

		\item En C, vous respecterez bien sûr la norme.\\

	\end{itemize}

    Bon courage !


\end{document}
%******************************************************************************%
